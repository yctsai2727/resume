%% start of file `template.tex'.
%% Copyright 2006-2013 Xavier Danaux (xdanaux@gmail.com).
%
% This work may be distributed and/or modified under the
% conditions of the LaTeX Project Public License version 1.3c,
% available at http://www.latex-project.org/lppl/.


\documentclass[10pt,a4paper,roman]{moderncv}        % possible options include font size ('10pt', '11pt' and '12pt'), paper size ('a4paper', 'letterpaper', 'a5paper', 'legalpaper', 'executivepaper' and 'landscape') and font family ('sans' and 'roman')

% modern themes
\moderncvstyle{banking}                            % style options are 'casual' (default), 'classic', 'oldstyle' and 'banking'
\moderncvcolor{blue}                                % color options 'blue' (default), 'orange', 'green', 'red', 'purple', 'grey' and 'black'
\renewcommand{\familydefault}{\sfdefault}         % to set the default font; use '\sfdefault' for the default sans serif font, '\rmdefault' for the default roman one, or any tex font name
\nopagenumbers{}                                  % uncomment to suppress automatic page numbering for CVs longer than one page

% character encoding
\usepackage[utf8]{inputenc}
\usepackage{fontawesome5}
%\usepackage{fontspec}
\usepackage{tabularx}
\usepackage{ragged2e}
% if you are not using xelatex ou lualatex, replace by the encoding you are using
%\usepackage{CJKutf8}                              % if you need to use CJK to typeset your resume in Chinese, Japanese or Korean

% adjust the page margins
\usepackage[scale=0.9,bottom=3pt,top=10pt]{geometry}
\usepackage{multicol}
%\setlength{\hintscolumnwidth}{3cm}                % if you want to change the width of the column with the dates
%\setlength{\makecvtitlenamewidth}{10cm}           % for the 'classic' style, if you want to force the width allocated to your name and avoid line breaks. be careful though, the length is normally calculated to avoid any overlap with your personal info; use this at your own typographical risks...

\usepackage{import}

\newcommand{\cvdoublecolumn}[2]{%
  \cvitem[0.75em]{}{%
    \begin{minipage}[t]{\listdoubleitemcolumnwidth}#1\end{minipage}%
    \hfill%
    \begin{minipage}[t]{\listdoubleitemcolumnwidth}#2\end{minipage}%
    }%
}
\newcommand{\cvreference}[5]{%
    \textbf{#1}\hfill%
    \ifthenelse{\equal{#5}{}}{}{\emailsymbol~\texttt{#5}\newline}% Name
    \ifthenelse{\equal{#2}{}}{}{\addresssymbol~#2\newline}%
    \ifthenelse{\equal{#3}{}}{}{#3\newline}%
    \ifthenelse{\equal{#4}{}}{}{#4}%
}

% personal data
\name{Yun Chen}{TSAI}

\newcommand*{\customcvedu}[7][.25em]{
    \begin{tabular}{@{}l} 
        {\bfseries #4}
      \end{tabular}
      \hfill% move it to the right
      \begin{tabular}{l@{}}
         {\bfseries #5}
      \end{tabular} \\
      \begin{tabular}{@{}l} 
        {\itshape #3}
      \end{tabular}
      \hfill% move it to the right
      \begin{tabular}{l@{}}
         {\itshape #2}
      \end{tabular}
      \ifx&#7&%
      \else{\\%
        \begin{minipage}{\maincolumnwidth}%
          \small#7%
        \end{minipage}}\fi%
      \par\addvspace{#1}
}
  
\newcommand*{\customcventry}[5][.25em]{
  \begin{tabular}{@{}l} 
    {\bfseries #2}
  \end{tabular}
  \hfill% move it to the right
  \begin{tabular}{l@{}}
     {\bfseries #3}
  \end{tabular} 
  %\begin{tabular}{@{}l} 
  %  {\itshape #1}
  %\end{tabular}
  %\hfill% move it to the right
  %\begin{tabular}{l@{}}
  %   {\itshape #2}
  %\end{tabular}
  \ifx&#5&%
  \else{\\%
    \begin{minipage}{\maincolumnwidth}%
      \small#5%
    \end{minipage}}\fi%
  %\par\addvspace{#1}
  }

\newcommand*{\customcvproject}[4][.25em]{
%   \vfill\noindent
  \begin{tabular}{@{}l} 
    {\bfseries #2}
  \end{tabular}
  \hfill% move it to the right
  \begin{tabular}{l@{}}
     {\itshape #3}
  \end{tabular}
  \ifx&#4&%
  \else{\\%
    \begin{minipage}{\maincolumnwidth}%
      \small#4%
    \end{minipage}}\fi%
  \par\addvspace{#1}}

\setlength{\tabcolsep}{12pt}

\let\cleardoublepage=\clearpage

%----------------------------------------------------------------------------------
%            content
%----------------------------------------------------------------------------------
\begin{document}
%\begin{CJK*}{UTF8}{gbsn}                          % to typeset your resume in Chinese using CJK
%-----       resume       ---------------------------------------------------------
\makecvtitle
\vspace*{-10mm}

\begin{center}
\begin{tabular}{ c c c c }
 \emailsymbol yctsai@connect.ust.hk & \faGithub\enspace yctsai2727 & \faMobile\enspace +852-54418187\\  
\end{tabular}
\end{center}

\section{SUMMARY}
I am a Year 4 undergraduate student at HKUST, majoring in Applied Mathematics. Experienced in academic research and finished various research projects under the supervision of Professor Shing Yu LEUNG, Amir GOHARSHADY, and Jean-François RASKIN. Those research projects focus on different aspects like scientific computation, algorithm design, formal methods, and complexity theory. My research interest mainly falls in the large area of formal methods, automata theory, and complexity theory, with the combination of a wide range of mathematical tools including but not limited to differential equations, optimization techniques, probability theory, algebraic method, to develop efficient algorithms and construct proofs for correctness, convergence and hardness result in the above field.

\section{EDUCATION}
{\customcvedu{Expected Graduation: June 2024}{BSc in Applied Mathematics}{Hong Kong University of Science and Technology (HKUST)}{}{}{}}

\section{PUBLICATION}
\begin{itemize}
  \item G. K. Conrado, A. K. Goharshady, K. Kochekov, Y. C. Tsai, A. K. Zaher (2023).  \textit{Exploiting the Sparseness of Control-flow and Call Graphs for Efficient and On-demand Algebraic Program Analysis.} ACM Conference on Object-Oriented Programming, Systems, Languages, and Applications, OOPSLA 2023.
  \item Y. C. Tsai, S. Y. Leung (2023). \textit{Local Trajectory Variation Exponent (LTVE) for Visualizing Dynamical Systems.} Submitted to Communication in Computational Physics, CiCP.
\end{itemize}

% \section{COURSE TAKEN}

% Only selected courses are recorded.\\

% {\customcventry{MATH course}{HKUST}{}
% {\begin{itemize}
%   \item MATH 1023 Honor Calculus I \hfill             grade: A
%   \item MATH 1024 Honor Calculus II \hfill            grade: A-
%   \item MATH 2043 Honor Mathematical Analysis \hfill  grade: A-
%   \item MATH 2431 Honor Probability \hfill            grade: A-
%   \item MATH 3043 Honor Real Analysis \hfill          grade: A
% \end{itemize}
% }}
% \\

% {\customcventry{COMP course}{HKUST}{}
% {\begin{itemize}
%   \item COMP 2011 Programming in C++ \hfill           grade: A
%   \item COMP 2012 Object-Oriented Programming and Data Structures \hfill    grade: A+
%   \item COMP 2711H Honor Discrete Mathematics Tool \hfill grade: A+
%   %\item COMP 3711 Design and Analysis of Algorithms \hfill grade: B+
%   \item COMP 3721 Theory of computation \hfill grade: A
%   \item COMP 5711 Introduction to Advanced Algorithmic Techniques \hfill grade: A+
% \end{itemize}
% }}
% \\

% {\customcventry{research course}{HKUST}{}
% {\begin{itemize}
%   \item UROP 1100 Efficient Algorithm on Visualizing Dynamical Surface \hfill grade: P*
%   \item UROP 1100 Parameterized algorithm for static program analysis \hfill grade: P*
%   \item UROP 2100 Parameterized algorithm for static program analysis \hfill grade: P*
%   \item SCIE 3500 IRE research Project I \hfill grade: A+
% \end{itemize}
% }}

%\newpage
\section{RESEARCH PROJECT}
{\customcventry{Project title}{Supervisor}{}{}
}\enspace\\

{\customcventry{Efficient algorithm on Visualizing Dynamical Surface}{Dr. Shing Yu LEUNG}{}
{\begin{itemize}
  \item In this project, we aim to develop efficient methods for visualizing the Lagrangian Coherent Structure under different settings, with a particular focus on the Lagrangian coherent structure. Various approaches like Finite-time Lyapunov Exponent, clustering methods, and trajectory analysis have been studied. In addition, we also look into the stochastic setting and study the extension of previous approaches.
\end{itemize}
}
}\enspace\\

{\customcventry{Parameterized Algorithms in Static Program Analysis}{Dr. Amir GOHARSHADY}{}
{\begin{itemize}
  \item This project is focusing on improving the existing algorithm for static program analysis through parameterization. The major motivation behind this is that many of the problems in the field are proven to be at least NP-hard in general, but the practical use case might exhibit special structures or carry small parameters. We look into problems like Algebraic program analysis and probabilistic model checking to look for potentially useful parameterization for the problems.
\end{itemize}
}
}\enspace\\

{\customcventry{Automata Learning and Program Synthesis}{Dr. Jean-François RASKIN}{}
{\begin{itemize}
  \item This project aims to study the use of automata learning to program synthesis problems. In particular, we looked into the problems of example-guided synthesis, we focus on the case when a stochastic environment input is involved and attempt to formulate a way to generate a set of optimal realizable examples of system trace with respect to a certain cost/reward function.
\end{itemize}
}}

\section{Scholarship}
\begin{itemize}
  \item Chern Class Talent Scholarship
  \item HKSAR Government Scholarship Fund - Reaching Out Award
\end{itemize}

\section{Reference}
Please contact the following professors for letter of recommendation.\\

\cvreference{Shing Yu LEUNG}
    {Professor, Department of Mathematics}
    {Hong Kong University of Science and Technology, Clear Water Bay, Hong Kong}
    {}
    {masyleung@ust.hk}

\vspace{-5pt}
\cvreference{Amir GOHARSHADY}
    {Assistant Professor, Department of Computer Science and Engineering}
    {Hong Kong University of Science and Technology, Clear Water Bay, Hong Kong}
    {}
    {goharshady@cse.ust.hk}

\vspace{-5pt}
\cvreference{Jean-François RASKIN}
    {Professor, Department of Computer Science}
    {Université Libre de Bruxelles, Bruxelles, Belgium}
    {}
    {jraskin@ulb.ac.be}\nolinebreak%
% \section{ADDITIONAL}
% \begin{minipage}{\maincolumnwidth}%
% 	\small{
%     	\begin{itemize}
%           \item Programming Languages: C, C++, Javascript, NodeJS, Python
% 		\end{itemize}}%
% \end{minipage}%
%-----       letter       ---------------------------------------------------------
\end{document}


%% end of file `template.tex'.